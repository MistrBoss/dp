% ============================================================================ %
% Encoding: UTF-8 (žluťoučký kůň úpěl ďábelšké ódy)
% ============================================================================ %

% ============================================================================ %
\nn{Úvod}
První odstavec pod nadpisem se neodsazuje, ostatní ano (pouze první řádek, odsazení vertikální mezy odstavci je typycké pro anglickou sazbu; czech babel toto respektuje, netřeba do textu přidávat jakékoliv explicitní formátování, viz ukázka sazby tohoto textu s následujícím odstavcem).

Formátování druhého odstavce. Text text text text text text text text text text text text.


% ============================================================================ %
\cast{Teoretická část}

\n{1}{Klasifikace řešených oblastí}
TODO: navodit téma + uvézt základní škálu

\n{2}{Konstruktivní změny fotografie}
TODO: definovat konstruktivní změny

\n{3}{Návrh řešení}
Křížová korelace zrychlená pomocí diskrétní Fourierovy transformace.
TODO: rozvést

\n{3}{Oblast zájmu}
TODO: uvézt

\n{4}{Změny vlastností}
\begin{itemize}
	\setlength{\parskip}{0pt}
	\setlength{\itemsep}{0pt}
	\setstretch{1.05}
	\item{Změna barev; TODO: Popis změny barev}
	\item{Změna kontrastu;TODO: Popis změny kontrastu}
	\item{Změna jasu;TODO: Popis změny jasu}
\end{itemize}

\n{4}{Změny obsahu}
\begin{itemize}
	\setlength{\parskip}{0pt}
	\setlength{\itemsep}{0pt}
	\setstretch{1.05}
	\item{Vodotisk}
	\item{Logo}
	\item{Šum}
\end{itemize}

\n{2}{Destruktivní změny fotografie}
TODO: definovat destruktivní změny

\n{3}{Návrh řešení}
Výpočet Hausdorfovy vzdálenosti mezi konvexními polyedry, které reprezentující hrany v obrazu.
TODO: rozvést

\n{3}{Oblast zájmu}
\begin{itemize}
	\setlength{\parskip}{0pt}
	\setlength{\itemsep}{0pt}
	\setstretch{1.05}
	\item{Změna komprese (rozmazaná fotka)}
	\item{Změny rozlišení}
	\begin{itemize}
		\item{Ořez (v jedné nebo obou dimenzích}
		\item{Deformace (v jedné nebo obou dimenzích)}
	\end{itemize}
\end{itemize}

\n{2}{Kombinace konstruktivních a destruktivních změn}
TODO: Obecně je porovnání problematické, rozvést.

\n{3}{Návrh řešení}
Redukce fotografie na její prahovou velikost jako příprava na křížovou korelaci viz konstruktivní změny.

\n{3}{Oblast zájmu}
\begin{itemize}
	\setlength{\parskip}{0pt}
	\setlength{\itemsep}{0pt}
	\setstretch{1.05}
	\item {Asymetrická změna obou stran s čímkoliv}
	\item {Změna kvality v důsledku zhoršeni komprese s čímkoliv}
	\item {Logo nebo vodotisk v kombinaci s předcházejícími}
\end{itemize}

\n{2}{Výběr reprezentativního vzorku}
Pro skupinu vzájemně si podobných fotografií vybereme nejvhodnějšího kandidáta, který bude následně ostatní fotografie zastupovat.

\n{3}{Návrh řešení}
Pro tyto účely zavedeme koeficient podobnosti, který zjednodušeně určíme jako VELIKOST × OSTROST + JAS. Výsledek bude na intervalu <0, 1>. Hodnoty blížící se nule mají nejnižší koeficient zastupitelnosti a naopak hodnoty blížící se 1 mají nejvyšší koeficient zastupitelnosti. 

\n{3}{Oblast zájmu}

\n{4}{Střední hodnota jasu}
Pouze zohlednění zda fotka není příliš jasná nebo příliš tmavá, jednoduchý algoritmus

\n{4}{Poměrný počet hran}
Test “rozmazanosti” fotky, konvoluce s vhodným (nutné testování) jádrem typu horní propust (s celkovým součtem 0) - tedy výsledek neutrální podklad z kterého “vystupují” hrany, velké množství hran ⇔ fotka není rozmazaná

\n{4}{Rozlišení fotografie}
Pouze klasická velikost fotky (větší ⇔ lepší).

\n{1}{Další nadpis}
Tato sekce obsahuje ukázku vložení obrázku (Obr. \ref{fig:logo}).

% Obrázek lze vkládat pomocí následujícího zjednodušeného stylu, nebo klasickým LaTex způsobem
% Pozor! Obrázek nesmí obsahovat alfa kanál (průhlednost). Jde to proti standardu PDF/A.
\obr{Popisek obrázku}{fig:logo}{0.5}{graphics/logo/fai_logo_cz.png}


\n{2}{Podnadpis}
Tato sekce obsahuje ukázku vložení tabulky (Tab. \ref{tab:priklad}).

% Tabulku lze vkládat pomocí následujícího zjednodušeného stylu, nebo klasickým LaTex způsobem
\tab{Popisek tabulky}{tab:priklad}{0.65}{|l|c|c|c|c|c|r|}{
  \hline
   & 1 & 2 & 3 & 4 & 5 & Cena [Kč] \\ \hline
  \emph{F} & (jedna) & (dva) & (tři) & (čtyři) & (pět) & 300 \\ \hline
}

\n{3}{Podpodnadpis}

\n{3}{Podpodnadpis}
Citace knihy. \cite{chmel}


% ============================================================================ %

% Pokud Vaše práce neobsahuje analytickou část, stačí odstranit či zakomentovat nasledujících pár rádků
\cast{Analytická část}

\n{1}{Brainstorming}
Analytický metoda sloužící ke sběru myšlenek, námětů a případné zevrubné konstruktivní kritice dané problematiky. V rámci této práce byla použita v akademickém a profesním kruhu pro identifikaci základních ukazatelů pro další kroky analýzy.

\n{2}{Akademický kruh}
Diskutovány zejména technické možnosti. Kde a jak lze vůbec porovnání fotografií provádět strojově. K další analýze vyly vyb

\n{2}{Profesní kruh}
V kruhu s provozovatelem byly kladeny nejvyšší nároky na flexibilitu a propustnost celého řešení.



\n{1}{Benchmarking}


% ============================================================================ %
\cast{Projektová část}

\n{1}{Distribuovaná služba}

\n{2}{Fronta nezpracovaných obrázků}

\n{2}{Servlet pro stažení obrázků}

\n{2}{Odeslání výsledků}

\n{1}{Klient}

% ============================================================================ %
\nn{Závěr}
Text závěru


% ============================================================================ %
